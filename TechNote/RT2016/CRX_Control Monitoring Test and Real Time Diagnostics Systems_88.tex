%% RT09_Instructions.tex
%% 4/2009
%% By Bo Yu (yu@bnl.gov)
%% based on:
%% bare_jrnl.tex
%% V1.3
%% 2007/01/11
%% by Michael Shell
%% see http://www.michaelshell.org/
%% for current contact information.
%%%%%%%%%%%%%%%%%%%%%%%%%%%%%%%%%%%%%%%%%%%%%%%%%%%%%%%%%%%%%%%%%
\documentclass[journal]{IEEEtran}
\pagestyle{empty}
\usepackage{graphicx}
\begin{document}
\title{Distributed and Parallel Real-time Control System \\ 
	Equipped FPGA-Zynq and EPICS Middleware}
%
% author names and IEEE memberships
% note positions of commas and nonbreaking spaces ( ~ ) LaTeX will not break
% a structure at a ~ so this keeps an author's name from being broken across
% two lines.
% use \thanks{} to gain access to the first footnote area
% a separate \thanks must be used for each paragraph as LaTeX2e's \thanks
% was not built to handle multiple paragraphs
%

\author{~Sangil Lee,
	~Changwook Son,
	and~Hyojae Jang% <-this % stops a space
\thanks{F. A. SI Lee is with the Institute for Basic Science, 70 Yuseong-daero, CO 34047 South Korea (telephone: +82-42-878-8854, e-mail: silee7103@ibs.re.kr.}%
\thanks{S. B. CW Son is with the Institute for Basic Science, 70 Yuseong-daero, CO 34047 South Korea (telephone: +82-42-878-8831, e-mail: scwook@ibs.re.kr.}%
\thanks{T. C. HY Jang is is with the Institute for Basic Science, 70 Yuseong-daero, CO 34047 South Korea (telephone: +82-42-878-8788, e-mail: lkcom@ibs.re.kr.}%
}

\maketitle
\thispagestyle{empty}

\begin{abstract}
Zynq series of Xilinx FPGA chips are divided into Processing System (PS) and Programmable Logic (PL), as a kind of SoC (System on Chip). PS with the dual-core ARM Cortex–A9 processor is performing the high-level control logic at run-time on linux operating system. PL with the low-level Field Programmable Gate Array (FPGA) built on high-performance, low-power, and high-k metal gate process technology is connecting with a lot of I/O peripherals for real-time control system. EPICS (Experimental Physics and Industrial Control System) is a set of open-source-based software tools which supports for the Ethernet-based middleware layer. In order to configure the environment of the distributed control system, EPICS middleware is equipped on the linux operating system of the Zynq PS. In addition, a lot of digital logic gates of the Zynq PL of FPGA-Zynq evaluation board (ZedBoard) are connected with I/O pins of the daughter board via FPGA Mezzanine Connector (FMC) of ZedBoard. An interface between the Zynq PS and PL is interconnected with AMBA4 AXI. For the organic connection both the PS and PL, it also used the linux device driver for AXI interface.
This paper describes the content and configuration of the distributed and parallel real-time control system applying FPGA-Zynq and EPICS middleware.
\end{abstract}

%\begin{IEEEkeywords}
%IEEEtran, journal, \LaTeX, paper, template.
%\end{IEEEkeywords}


\section{Introduction}
% The very first letter is a 2 line initial drop letter followed
% by the rest of the first word in caps.
% 
% form to use if the first word consists of a single letter:
% \IEEEPARstart{A}{demo} file is ....
% 
% form to use if you need the single drop letter followed by
% normal text (unknown if ever used by IEEE):
% \IEEEPARstart{A}{}demo file is ....
% 
% Some journals put the first two words in caps:
% \IEEEPARstart{T}{his demo} file is ....
% 
% Here we have the typical use of a "T" for an initial drop letter
% and "HIS" in caps to complete the first word.
\IEEEPARstart{R}{AON} is a new heavy ion accelerator under construction in South Korea, which is to produce a variety of stable ion and rare isotope beams to support various researches for the basic science and applied research applications\cite{risp}. RAON, which is a colossal machine, is composed of many control devices, experimental equipments, and additional utilities. In order to control a enormous machine such as a RAON, it needs to the middle-ware layer to support for the distributed control system. For the purpose of  implementing it, the control system of RAON has decided to apply the EPICS (Experimental Physics and Industrial Control System) software framework. "EPICS is a set of open source software tools, libraries and applications developed collaboratively and used worldwide to create distributed soft real-time control systems for scientific instruments such as a particle accelerators, telescopes and other large scientific experiments"\cite{epics}. Additionally for monitoring or control system of the fast response time, FPGA chips are in general use.


%빠른 응답성을 갖는 고속의 제어 또는 모니터링 시스템을 위하여, FPGA 칩이 사용된다.

you may submit your manuscript to the \emph{IEEE Transactions on Nuclear Science (TNS)} if it represents significant original contributions in the fields associated with the conference (i.e., progress reports and preliminary findings are not appropriate).  The TNS is a refereed publication, and is published throughout the year. The reviewing process, independent from the conference, will be performed during the weeks after the conference. For instructions on TNS  manuscript submissions, please visit the IEEE's on-line peer review system Manuscript Centra   (http://mc.manuscriptcentral.com/tns-ieee). Please note that submission to TNS  is a separate process from that of the Conference Record.

\section{Procedure for Manuscript Submission}

The manuscript submission is done online on the paper submission web site.  See details below. All questions regarding CR submission should be directed to Dora Merelli, the Guest Editor, at dora.merelli@ieee.fr. Formatting and submission assistance will be provided at the conference in the Guest Editor room. We ask all of you to pass by with your paper before the end of the conference. This will allow the Guest Editor to check the format of your paper and see the possible correction and modification directly on site.

\subsection{Create IEEE Xplore-Compatible PDF File}

All manuscripts submitted to IEEE for publication must meet the PDF Specification for IEEE Xplore\cite{IEEEPDFRequirement401}. To assist authors in meeting this requirement, IEEE has established a web based service called PDF eXpress. You can use this web service to convert your dvi  files into Xplore-compatible PDF files, or to check if your own PDF file is Xplore-compatible. You must put your dvi and all graphics files into a compressed archive to upload to PDF eXpress.

The PDF eXpress service will be available to the RT14 authors. To use this service, you will need to go to http://www.pdf-express.org, and enter the Conference ID, which is \textbf{31098X}. If you are a first time user of this system, you need to set up an account.  Once logged in, follow the instructions on the web site to upload your word processor file or PDF file.  Shortly after your file is uploaded to the PDF eXpress, you will receive an email. If you uploaded a word processor file for conversion, the attachment in this email will be the converted Xplore-compatible PDF file.  Save this file for the submission step outlined in section II.B below.  If you uploaded a PDF file for checking, the email will show if your file passed or failed the check.  If your PDF file failed the check, read the error report and fix the identified problem(s).  Re-upload your PDF file and have it checked again until your PDF file is Xplore-compatible.

You can also bypass the PDF eXpress service and create your own Xplore-complatible PDF files.  The key requirements are the following:
\begin{itemize}
\item[1]	Do not protect your PDF file with any password;
\item[2]	Embed all fonts used in the document;
\item[3]	Do not embed any bookmarks or hyperlinks.
\end{itemize}


Use Type 1 postscript fonts when converting dvi file into postscript.  If you are using older versions of dvips to generate the postscript file, you might need to include -P pdf in the command line to produce postscript file optimized for distilling to PDF.

A detailed description of the IEEE Xplore-compatible PDF requirement is available at http://www.ieee.org/documents/31296\_IEEE\_PDF\_Spec.zip \cite{IEEEPDFRequirement401}.  
If you are using a Windows version of the Adobe Distiller to create PDF files, you can download a set of job option files from http://www.ieee.org/documents/IEEE\_PDF\_Create.zip. Install and use the appropriate job option to create your own Xplore-compatible PDF files.  
If you are using other software packages to generate PDF files, please refer to their manuals for correct conversion settings.  The most common problem in creating Xplore-compatible PDF files is not embedding all fonts.
 


% An example of a floating figure using the graphicx package.
% Note that \label must occur AFTER (or within) \caption.
% For figures, \caption should occur after the \includegraphics.
% Note that IEEEtran v1.7 and later has special internal code that
% is designed to preserve the operation of \label within \caption
% even when the captionsoff option is in effect. However, because
% of issues like this, it may be the safest practice to put all your
% \label just after \caption rather than within \caption{}.
%
% Reminder: the "draftcls" or "draftclsnofoot", not "draft", class
% option should be used if it is desired that the figures are to be
% displayed while in draft mode.
%
%\begin{figure}[!t]
%\centering
%\includegraphics[width=3.5in]{myFigure.eps}
% where an .eps filename suffix will be assumed under latex, 
% and a .pdf suffix will be assumed for pdflatex; or what has been declared
% via \DeclareGraphicsExtensions.
%\caption{Daily abstract submission rate of the 2007 NSS-MIC. }
%\label{fig_sim}
%\end{figure}

% Note that IEEE typically puts floats only at the top, even when this
% results in a large percentage of a column being occupied by floats.


% An example of a double column floating figure using two subfigures.
% (The subfig.sty package must be loaded for this to work.)
% The subfigure \label commands are set within each subfloat command, the
% \label for the overall figure must come after \caption.
% \hfil must be used as a separator to get equal spacing.
% The subfigure.sty package works much the same way, except \subfigure is
% used instead of \subfloat.
%
%\begin{figure*}[!t]
%\centerline{\subfloat[Case I]\includegraphics[width=2.5in]{subfigcase1}%
%\label{fig_first_case}}
%\hfil
%\subfloat[Case II]{\includegraphics[width=2.5in]{subfigcase2}%
%\label{fig_second_case}}}
%\caption{Simulation results}
%\label{fig_sim}
%\end{figure*}
%
% Note that often IEEE papers with subfigures do not employ subfigure
% captions (using the optional argument to \subfloat), but instead will
% reference/describe all of them (a), (b), etc., within the main caption.


% An example of a floating table. Note that, for IEEE style tables, the 
% \caption command should come BEFORE the table. Table text will default to
% \footnotesize as IEEE normally uses this smaller font for tables.
% The \label must come after \caption as always.
%
%\begin{table}[!t]
%% increase table row spacing, adjust to taste
%\renewcommand{\arraystretch}{1.3}
% if using array.sty, it might be a good idea to tweak the value of
% \extrarowheight as needed to properly center the text within the cells
%\caption{An Example of a Table}
%\label{table_example}
%\centering
%% Some packages, such as MDW tools, offer better commands for making tables
%% than the plain LaTeX2e tabular which is used here.
%\begin{tabular}{|c||c|}
%\hline
%One & Two\\
%\hline
%Three & Four\\
%\hline
%\end{tabular}
%\end{table}


% Note that IEEE does not put floats in the very first column - or typically
% anywhere on the first page for that matter. Also, in-text middle ("here")
% positioning is not used. Most IEEE journals use top floats exclusively.
% Note that, LaTeX2e, unlike IEEE journals, places footnotes above bottom
% floats. This can be corrected via the \fnbelowfloat command of the
% stfloats package.



% if have a single appendix:
%\appendix[Proof of the Zonklar Equations]
% or
%\appendix  % for no appendix heading
% do not use \section anymore after \appendix, only \section*
% is possibly needed

% use appendices with more than one appendix
% then use \section to start each appendix
% you must declare a \section before using any
% \subsection or using \label (\appendices by itself
% starts a section numbered zero.)
%


\subsection{Submit the Manuscript and Copyright Form}

After you have obtained the Xplore-compatible PDF file, you have to submit it to the paper submission web site. Be aware that PDF files that are not Xplore-compatible will not be included in the Conference Record CD.


An IEEE Copyright Form should be submitted electronically at the same time your Xplore-compatible manuscript is submitted. Details on how to submit an IEEE copyright form will be given on the Conference web site.  Each manuscript submitted to the Conference Record must be accompanied by a corresponding copyright form. 

\newpage


\appendices
\section{}

The {\LaTeX} class file included in the above package is the current version of the IEEEtran.cls (version 1.7).  Use the "{\bf journal}" mode to format your document so that the page layout follows the standard style for the IEEE Transactions. However, you must include these two commands in your source file to prevent the generation of headers and footers in the manuscript:
\begin{itemize}
\item add the command \verb1\pagestyle{empty}1 right after  \verb1\documentclass1; 
\item add the command \verb1\thispagestyle{empty}1 right after \verb1\maketitle1.  
\end{itemize}

Do not include your biography information at the end of the manuscript.  



% use section* for acknowledgement
\section*{Acknowledgment}
This work is supported by the Rare Isotope Science Project funded by Ministry of Science, ICT and Future Planning {(MSIP)} and National Research Foundation {(NRF)} of Korea (Project No. 2011-0032011).

% references section

% can use a bibliography generated by BibTeX as a .bbl file
% BibTeX documentation can be easily obtained at:
% http://www.ctan.org/tex-archive/biblio/bibtex/contrib/doc/
% The IEEEtran BibTeX style support page is at:
% http://www.michaelshell.org/tex/ieeetran/bibtex/
%\bibliographystyle{IEEEtran}
% argument is your BibTeX string definitions and bibliography database(s)
%\bibliography{IEEEabrv,../bib/paper}
%
% <OR> manually copy in the resultant .bbl file
% set second argument of \begin to the number of references
% (used to reserve space for the reference number labels box)
\begin{thebibliography}{3}

\bibitem{risp} Y.~K.~Kwon, {\it et. al},``Status of Rare Isotope Science Project in Korea'',
Few-Body Syst 54, 961-966, (2013).


\bibitem{epics}
EPICS website, http://www.aps.anl.gov/epics/

\bibitem{IEEEhowto:kopka}
H.~Kopka and P.~W. Daly, \emph{A Guide to \LaTeX}, 3rd~ed.\hskip 1em plus
  0.5em minus 0.4em\relax Harlow, England: Addison-Wesley, 1999.

\bibitem{IEEEPDFRequirement401}
IEEE Content Engineering, \emph{IEEE PDF Specification Version 4.10}. Available: http://www.ieee.org/documents/31296\_IEEE\_PDF\_Spec.zip.


\end{thebibliography}




% that's all folks
\end{document}


