%% RT09_Instructions.tex
%% 4/2009
%% By Bo Yu (yu@bnl.gov)
%% based on:
%% bare_jrnl.tex
%% V1.3
%% 2007/01/11
%% by Michael Shell
%% see http://www.michaelshell.org/
%% for current contact information.
%%%%%%%%%%%%%%%%%%%%%%%%%%%%%%%%%%%%%%%%%%%%%%%%%%%%%%%%%%%%%%%%%
\documentclass[journal]{IEEEtran}
\pagestyle{empty}
\usepackage{graphicx}
\begin{document}
\title{Distributed and Parallel Real-time Control System \\ 
	Equipped FPGA-Zynq and EPICS Middleware}
%
% author names and IEEE memberships
% note positions of commas and nonbreaking spaces ( ~ ) LaTeX will not break
% a structure at a ~ so this keeps an author's name from being broken across
% two lines.
% use \thanks{} to gain access to the first footnote area
% a separate \thanks must be used for each paragraph as LaTeX2e's \thanks
% was not built to handle multiple paragraphs
%

\author{~Sangil Lee,
	~Changwook Son,
	and~Hyojae Jang% <-this % stops a space
\thanks{F. A. SI Lee is with the Institute for Basic Science, 70 Yuseong-daero, CO 34047 South Korea (telephone: +82-42-878-8854, e-mail: silee7103@ibs.re.kr.}%
\thanks{S. B. CW Son is with the Institute for Basic Science, 70 Yuseong-daero, CO 34047 South Korea (telephone: +82-42-878-8831, e-mail: scwook@ibs.re.kr.}%
\thanks{T. C. HY Jang is is with the Institute for Basic Science, 70 Yuseong-daero, CO 34047 South Korea (telephone: +82-42-878-8788, e-mail: lkcom@ibs.re.kr.}%
}

\maketitle
\thispagestyle{empty}

\begin{abstract}
Zynq series of Xilinx FPGA chips are divided into Processing System (PS) and Programmable Logic (PL), as a kind of SoC (System on Chip). PS with the dual-core ARM Cortex–A9 processor is performing the high-level control logic at run-time on linux operating system. PL with the low-level Field Programmable Gate Array (FPGA) built on high-performance, low-power, and high-k metal gate process technology is connecting with a lot of I/O peripherals for real-time control system. EPICS (Experimental Physics and Industrial Control System) is a set of open-source-based software tools which supports for the Ethernet-based middleware layer. In order to configure the environment of the distributed control system, EPICS middleware is equipped on the linux operating system of the Zynq PS. In addition, a lot of digital logic gates of the Zynq PL of FPGA-Zynq evaluation board (ZedBoard) are connected with I/O pins of the daughter board via FPGA Mezzanine Connector (FMC) of ZedBoard. An interface between the Zynq PS and PL is interconnected with AMBA4 AXI. For the organic connection both the PS and PL, it also used the linux device driver for AXI interface.
This paper describes the content and configuration of the distributed and parallel real-time control system applying FPGA-Zynq and EPICS middleware.
\end{abstract}

%\begin{IEEEkeywords}
%IEEEtran, journal, \LaTeX, paper, template.
%\end{IEEEkeywords}


\section{Introduction}
% The very first letter is a 2 line initial drop letter followed
% by the rest of the first word in caps.
% 
% form to use if the first word consists of a single letter:
% \IEEEPARstart{A}{demo} file is ....
% 
% form to use if you need the single drop letter followed by
% normal text (unknown if ever used by IEEE):
% \IEEEPARstart{A}{}demo file is ....
% 
% Some journals put the first two words in caps:
% \IEEEPARstart{T}{his demo} file is ....
% 
% Here we have the typical use of a "T" for an initial drop letter
% and "HIS" in caps to complete the first word.
\IEEEPARstart{R}{AON} is a new heavy ion accelerator under construction in South Korea, which is to produce a variety of stable ion and rare isotope beams to support various researches for the basic science and applied research applications\cite{risp}. RAON, which is a colossal machine, is composed of many control devices, experimental equipments, and additional utilities. A big scientific experiment is required by the unity of the heterogenous distributed control system according to its characteristic and the fast response for a particular device. Furthermore, these requirements for the distributed and parallel environments must be satisfied at the same time. The distributed processing is to handle dividing one job into several processes and the parallel processing is to handle the multiple jobs at the same time. 
To satisfy the general requirements under the distributed environment, it needs to the middle-ware layer to support for the distributed control system. For the purpose of  implementing it, the control system of RAON has decided to apply the EPICS (Experimental Physics and Industrial Control System) software framework. "EPICS is a set of open source software tools, libraries and applications developed collaboratively and used worldwide to create distributed soft real-time control systems for scientific instruments such as a particle accelerators, telescopes and other large scientific experiments"\cite{epics}. Additionally FPGA chips are used in general for monitoring or control system of the fast response time. There are also many control systems that adopted FPGA among the RAON equipments, such as timing system, LLRF control system, power supply cotrol system and so on. A lot of devices or systems have implemented the control system configuring CPUs with OS and FPGA chips using VME or cPCI interface. The low-power clustering application for parallel computing is also one of the fields utilizing the GPU or FPGA. Likewise, the configuration of the specific hardwares, such as GPU processing elements or FPGA gate logics, it can be saved the prorated cost of the CPU for control and analysis algorithms. Zynq series of Xilinx FPGA chips are called the programmable SoC. Within the one chip it consists of the arm-based CPU and the logic gates of FPGA. PS with the dual-core ARM Cortex–A9 processor is performing the high-level control logic at run-time on linux operating system. PL with the low-level FPGA is connecting with a lot of I/O peripherals for real-time control system. Utilizing Zynq SoC, it could configure an environment of the distributed and parallel processing at a time. In addition, I guess that the control system may provide an abstract layer of the parallel processing integrated with EPICS middleware using Open Computing Lanaguage (OpenCL)\cite{opencl} for the specific heterogenous hardwares in future plan.

\section{Distributed Processing Environment}
Configuration of Fig.\ref{control_system} shows the overall of the control system for the distributed and parallel processing. The PS of zynq is equipped with the EPICS software framework on the cross-compiled linux OS of the ARM processor. The parameter values for the motor control is set through the EPICS interface on PS. The data communication between PS and PL uses the AXI interface of the ARM Advanced Microcontroller Bus Architecture (AMBA). Software module for AXI interface between both should be also developed in the linux device driver. The types of motor which can be supported by a low voltage drive board of the Analog Device\cite{analog} are brushless DC, PMSM, brushed DC or stepper motor.

\begin{figure*}[!tbh]
	\centering
	\includegraphics*[width=\textwidth,height=0.6\textwidth]{img01.png}
	\caption{Distributed and Parallel Control System Configuration}
	\label{control_system}
\end{figure*}

\subsection{Linux on Zynq PS}
To operate linux OS on the ARM processor of the zynq PS there consists of:
\begin{itemize}
	\item ARM Cross Compile Tool Chain (arm-linux-gnueabihf)
	\item Linux Kernel Source (Linaro)
	\item Bootloader (BOOT.BIN)
	\item Board Support Package (Linux Device Tree)
	\item Root File System (Busybox)
\end{itemize}

Linaro\cite{linaro} is a not-for-profit engineering organization that works on free and open-source software such as the Linux kernel, the GNU Compiler Collection (GCC), power management, graphics and multimedia interfaces for the ARM family of instruction sets and implementations thereof as well as for the heterogeneous system architecture. The linaro kernel source is the ubuntu-based kernel source including a lot of device driver such as zynq device tree. 
The bootloader of PS uses U-Boot\cite{u-boot} supporting for the zynq device of Xilinx. To boot the zynq device, it should be made up the boot.bin file created by Vivado\cite{vivado} tool. The boot.bin\cite{boot-bin} file consists of fsbl.elf (First Stage BootLoader), u-boot.elf (Second BootLoader), uImage (kernel zImage for uboot ), zynq.bif (Boot Image Format file) and user.bit (bitstream file of FPGA). The linux kernel is using the root file system generated by the Busybox\cite{busybox} tool.


\subsection{Device Driver on Linux}
Linux device driver for zynq PS uses Industrial IO (IIO) module of the Analog Devices. Libiio\cite{iio} is a library that has been developed by Analog Devices for easy interface with IIO devices. Localbackend module of Fig.~\ref{control_system} goes into kernel mode from user mode through the system call and accesses the iio device driver via "struct file\_operations". Basically the iio device driver is a character device type.

\subsection{EPICS Middleware Framework}
EPICS IOC is being implemented using the libiio library that abstracted the low-level details of the hardware, as shown in Fig.~\ref{control_system}. EPICS is a set of open source software tools, libraries and applications developed collaboratively and used worldwide to create distributed soft real-time control systems for scientific instruments such as a particle accelerators, telescopes and other large scientific experimental facilities\cite{epics}. The device support routine of EPICS is implemented using the libiio library to read or write the parameter values for the motor control. The motor control parameters become the Process Variables (PVs) of EPICS and are carried out the EPICS database processing according to EPICS scan rate. Also, it is going to develop additional EPICS waveform PV to monitor the PWM signal of the low voltage drive board.


\cleardoublepage
\section{Parallel Processing Environment}


\section{Software Defined System on Chip}


\section{Future Work}
\subsection{OpenCL for Parallel Processing}
OpenCL is a programming model developed by the Khronos group for the parallel processing on the specific hardwares like GPU and FPGA.


\begin{figure}[!htb]
	\centering
	\includegraphics*[width=75mm, height=50mm]{img02.png}
	\caption{OpenCL-based Parallel Control System Configuration}
	\label{opencl_control_system}
\end{figure}



\subsection{on GPU}


\subsection{on FPGA}

\begin{table}[h!t]
%	\setlength\tabcolsep{3.8pt}
\centering
\caption{Software Modules}
	\label{sw-conf}
	
	\begin{tabular}{@{}lll@{}}
		%\begin{tabular}{clc}}
		
		\hline
		\textbf{Software} & \textbf{Contents}                 & \textbf{Module}     \\
		\hline
		EPICS        & Base R3.14.15.2                       & PS of Zynq            \\
		Libiio       & Industrial I/O library                &                       \\
		& supplied Analog Devices               &                       \\         
		Kernel       & Linaro kernel source                  &                       \\
		& including iio device driver           &                       \\         
		Bootloader   & U-Boot for zynq                       &                       \\
		IOC          & In-house using Libiio                 &                       \\
		
		\hline
		FPGA         & Analog Devices and                    & PL of Zynq            \\
		& In-house code                         &                       \\
		
		\hline
		VxWorks      & Real-time OS for VME                  & Timing                \\
		& including BSP                         &                       \\         
		MRFIOC2      & EPICS IOC for timing                  &                       \\
		
		\hline
		SRSIOC       & IOC for rubidium clock                &                       \\
		
		\hline
		Workbench    & Workbench3.3 for VxWorks              & Development           \\
		Vivado       & Vivado14.4 including SDK              & Tool                  \\        
		& with node lock license                &                       \\            
		Busybox      & for rootfs system (free)              &                       \\
		Toolchain    & for ARM cross-compile (GNU)           &                       \\        
		\hline
		
	\end{tabular}
\end{table}


\subsection{EPICS v4 PV Access}

\newpage

\section{Conclusion}
%가속기, 핵융합, 천체 망원경과 같은 거대 과학 실험 장치는 그 특성상 분산된 이기종 제어 장치들의 통일성이 요구되며, 특정 장치에 대해서는 빠른 응답성이 요구된다. 더욱이 이 두가지 요구사항은 동시에 만족되어야 한다.TCP/UDP/IP protocol Ethernet-based EPICS middleware framework는 여러 이기종 제어장치들을 하나의 단일 제어시스템으로 구현할 수 있는 환경을 제공한다.
% 또한 향 후에는 OpenCL를 이용하여 특정 하드웨어들에 대한 병렬처리 부분에 대한 추상화 layer를 제공하고 EPICS와 통합할 수 있을 것으로 생각된다.

Timing system can distribute the fine synchronized event signal at a high speed. The objective of the stepper motor control testbed is to know how to operate the timing system and how to apply it to the high speed controller. The overall implementation is still underway, however if the distributed control system using EPICS makes a connection with the high speed parallel processing of FPGA, it is possible to improve the performance and efficiency of the control system. Zynq SoC can be considered as an ideal device to implement this high-speed control system.

Utilizing the OpenCL standard on an FPGA may offer significantly higher performance and at much lower power than is available today from hardware architectures (CPU, GPUs, etc). In addition, an FPGA-based heterogeneous system (CPU + FPGA) using the OpenCL standard has a significant time-to-market advantage compared to traditional FPGA development using lower level hardware description languages (HDLs) such as Verilog or VHDL. Altera joined The Khronos Group in 2010 and is an active contributor to the standard. To stay up to date on Altera’s OpenCL program for FPGAs, please register at www.altera.com/opencl.

% use section* for acknowledgement
\section*{Acknowledgment}
This work is supported by the Rare Isotope Science Project funded by Ministry of Science, ICT and Future Planning \textbf{(MSIP)} and National Research Foundation \textbf{(NRF)} of Korea (Project No. 2011-0032011).


% references section

% can use a bibliography generated by BibTeX as a .bbl file
% BibTeX documentation can be easily obtained at:
% http://www.ctan.org/tex-archive/biblio/bibtex/contrib/doc/
% The IEEEtran BibTeX style support page is at:
% http://www.michaelshell.org/tex/ieeetran/bibtex/
%\bibliographystyle{IEEEtran}
% argument is your BibTeX string definitions and bibliography database(s)
%\bibliography{IEEEabrv,../bib/paper}
%
% <OR> manually copy in the resultant .bbl file
% set second argument of \begin to the number of references
% (used to reserve space for the reference number labels box)
\begin{thebibliography}{3}

\bibitem{risp} Y.~K.~Kwon, {\it et. al},``Status of Rare Isotope Science Project in Korea'',
Few-Body Syst 54, 961-966, (2013).


\bibitem{epics}
EPICS website, http://www.aps.anl.gov/epics/

\bibitem{IEEEhowto:kopka}
H.~Kopka and P.~W. Daly, \emph{A Guide to \LaTeX}, 3rd~ed.\hskip 1em plus
  0.5em minus 0.4em\relax Harlow, England: Addison-Wesley, 1999.

\bibitem{IEEEPDFRequirement401}
IEEE Content Engineering, \emph{IEEE PDF Specification Version 4.10}. Available: http://www.ieee.org/documents/31296\_IEEE\_PDF\_Spec.zip.

\bibitem{opencl}
OpenCL, https://www.khronos.org/opencl/

\bibitem{linaro}
Linaro Document website, https://en.wikipedia.org/wiki/Linaro

\bibitem{opencl_on_altera}
OpenCL on Altera, https://www.altera.com/en\_US/pdfs/literature/wp/wp-01173-opencl.pdf

\bibitem{analog}
Analog Devices website, http://www.analog.com


\bibitem{iio}
Industrial I/O Document website, https://wiki.analog.com/resources/tools-software/linux-software/libiio


\end{thebibliography}




% that's all folks
\end{document}


