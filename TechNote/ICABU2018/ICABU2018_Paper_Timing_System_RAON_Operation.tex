%% RT09_Instructions.tex
%% 4/2009
%% By Bo Yu (yu@bnl.gov)
%% based on:
%% bare_jrnl.tex
%% V1.3
%% 2007/01/11
%% by Michael Shell
%% see http://www.michaelshell.org/
%% for current contact information.
%%%%%%%%%%%%%%%%%%%%%%%%%%%%%%%%%%%%%%%%%%%%%%%%%%%%%%%%%%%%%%%%%
\documentclass[journal,reqno]{IEEEtran}

\pagestyle{empty}
\usepackage{graphicx}

\usepackage{booktabs}
\usepackage{verbatim}
\usepackage{framed}

\usepackage{xcolor}
\usepackage[utf]{kotex}
\usepackage{ifpdf}
\usepackage[unicode,bookmarks]{hyperref}
\usepackage{alltt}
\usepackage{authblk}
\usepackage{caption}
\usepackage{amsmath}

%\setlength{\textwidth}{425pt}
%\setlength{\hoffset}{-40pt}

%\parindent0em

\newcommand\SEC[1]{\textbf{\uppercase{#1}}}
\begin{document}
\title{
	\textbf{Timing System Development for RAON Operation} %\\ 
	\vspace{0.5cm}
	\Large
}
\author[1]{Sangil Lee\thanks{silee7103@ibs.re.kr}}
\author[2]{Changwook Son\thanks{scwook@ibs.re.kr}}
\affil[1,2]{Accelerator Engineering Team, Institute for Basic Science, Daejeon 34047}
\author[3]{Seung Yong Kim\thanks{laykim@durutronix.co.kr}}
\affil[3]{Durutronix, Daejeon 34036}
\author[4]{Cheol-Hoon Lee\thanks{clee@cnu.ac.kr}}
\affil[4]{Corresponding author, Chungnam National University. Daejeon 34047}
\date{}
\maketitle
\vspace{2em}

\begin{abstract}
RAON (Rare Isotope Accelerator complex for ON-line experiments) is a heavy-ion accelerator experiment facility under construction in Korea. This facility aims at beam energy of 200MeV/u and maximum beam power of 400kW. Also, it is expected to be completed by 2021. The large-scaled experiment facility like RAON requires the timing system for the precise-synchronized operation. For this purpose, domestic accelerator facilities mainly use the timing system of foreign products. The accelerator engineering team of RAON has successfully developed the RAON timing system in cooperation with the domestic company. The signals of RAON timing system synchronized with RF reference clock and GPS are distributed the whole RAON site through the dedicated timing network (3.25 Gbps). The time accuracy of the timing system is 12.3ns synchronized with the RF reference clock (81.25 MHz). All signals (triggers, pulse delayed clock, and external signals) of timing system developed using the EPICS (Experimental Physics and Industrial Control System) software module are configurable as well software definition. In addition to, the RAON timing system supports the more flexible system for the various beam operation mode and improves the performance of the control system. \newline
This paper describes the results of the design and development for the RAON specific timing system.
\end{abstract}

%\begin{IEEEkeywords}
%IEEEtran, journal, \LaTeX, paper, template.
%\end{IEEEkeywords}


\section{Introduction}
% The very first letter is a 2 line initial drop letter followed
% by the rest of the first word in caps.
% 
% form to use if the first word consists of a single letter:
% \IEEEPARstart{A}{demo} file is ....
% 
% form to use if you need the single drop letter followed by
% normal text (unknown if ever used by IEEE):
% \IEEEPARstart{A}{}demo file is ....
% 
% Some journals put the first two words in caps:
% \IEEEPARstart{T}{his demo} file is ....
% 
% Here we have the typical use of a "T" for an initial drop letter
% and "HIS" in caps to complete the first word.

\IEEEPARstart{R}{AON} is a new heavy ion accelerator under construction in South Korea, which is to produce a variety of stable ion and rare isotope beams to support various researches for the basic science and applied research applications\cite{risp}. The RAON is the important facility of IBS and aims to realize the core infrastructure of the next generation basic science research. To control the large-scaled facility like RAON, the timing system plays a role as one of the important control systems. There are two ways of RAON operation. One is a continuous wave operation and the other is a pulse width modulation operation. The RAON timing system is related to the pulse mode operation. The RAON timing system is a system that precisely synchronizes the local devices (LLRF, Chopper, MPS, Beam Diagnostics and so on) of the whole accelerator. Hardware event signals of the timing system synchronized from GPS must be distributed to each local control system according to various operation and beam mode for RAON operation. It also provides accurate time information through all the site as a time source.

\section{Requirements}
The timing system consists of three sub-modules. One is the event generator (EVG) module, another is the event fan-out (EVF) module and the other is the event receiver (EVR) module.
The requirements for developing the RAON timing system are as follows:

\begin{itemize}
	\item Sub-modules (EVG, EVR, EVF) of the RAON timing system must be fully integrated on the EPICS middleware for interoperation with higher control systems.
	\item Sub-modules have to provide for the health status with themselves.
	\item Clock jitter must be less than 1ns and timing jitter must be less than 50ps.
	\item EVG of the timing system must be able to take a sequence of events as input from control system. Time of occurrence for each event in sequence is pre-defined.
	\item Granularity of EVG event emission in a timing sequence Tevent (1/Fevent) = 12.31ns. Fevent is equal to fundamental RF reference clock of 81.25MHz.
	\item EVG of the timing system has to repeat the same timing sequence till receiving the new sequence (repetition rate: 1 – 100 Hz).
	\item EVG of the timing system has to emit arbitrary event on user request without the need to run a sequence.
	\item Granularity of EVR for setting event delay and width parameters of responses generated on hardware output signals must be Tevent (=12.31ns). 
	\item EVR of the timing system has to output synchronized timing system clock with configurable integer divider.
	\item EVR of the timing system has to provide a timestamp for all configured and received timing signals.
	\item All EVRs of the timing system has fallback mode to switch to local clock source for losing the recovery clock from the timing master clock (RF reference clock).
	\item EVF of the timing system must fan out the timing event stream by an EVG to an array of EVRs through the timing network (optical link).
	\item EVF of the timing system measures the compensation delay time between each node.
\end{itemize}

\begin{figure*}[!htb]
	\centering
	%	\captionsetup{justifcation=centering}
	%	\includegraphics*[width=1\textwidth, height=0.6\textwidth]{img01.png}
	%	\caption{RAON Integrated Control System Overview}
	\includegraphics*[width=1\textwidth, height=0.5\textwidth]{img1-1.png}
	%	\includegraphics*[width=0.7\textwidth, height=1\textwidth, angle=90]{img1-2.png}
	\caption{Timing Data Structure}
	\label{timing_data_structure}
\end{figure*}

%\cleardoublepage
\section{Development}
The RAON timing system was developed using Xilinx’s programmable SoC (Zynq) with built-in ARM processor and FPGA logic. Zynq of Xilinx FPGA series is divided into Programmable Logic (PL) and Processing System (PS). PS with the dual-core ARM CortexA9 processor is performing the high-level control logic at run-time on linux operating system. PL with the low-level Field Programmable Gate Array (FPGA) is connecting with a lot of I/O peripherals for real-time control system. In order to set up the timing parameters connected with the higher control system, EPICS (Experimental Physics and Industrial Control System) middleware is equipped on the linux operating system of the Zynq PS. EPICS is a set of open-source-based software tools which supports for the Ethernet-based middleware layer. An interface between the Zynq PS and PL is interconnected with AMBA4 AXI. The event generator (EVG) of RAON timing system receives 16 TTL inputs, generates event codes and sends those to the specific timing optical network. The event fan-out (EVF) of the timing system provides the ability to distribute up to 24 ports (SFP fiber optical) on the timing network. The event receiver (EVR) should be synchronized with the event generator and provides the capability to generate up to 32 TTL pulses. 

\subsection{Timing Structure Design}
The timing data (32bit) is transmitted with an external clock of 81.25 MHz (RF reference clock). The timing data consists of Stream data (8bit), DBus (Distributed Bus 8bit), evCodeA (Event Code A 5bit), and evCodeB (Event Code B 11 bit). In Fig. 1, the timing operating clock of all equipment connected to the timing network is synchronized with the EVG external reference clock. A total signal sources (evSrc 31ea) generating in EVG are composed of external inputs (exIn 16ea), MXCs (Multiplexed Counter 14ea) and user trigger. Stream data of 8 bits includes the header, K-code, timestamp, event status, and tail. DBus data of 8 bits is 8 signals selected from inputs of EVG. Event code A of 5 bits is an event from 1 to 32 generated by selected 32 inputs from EVG. Event code B of 11 bits is an event generated in sequence A and B by the signals combining inputs of EVG. Event code A can also be mapped to event code B of 11 bits.

\begin{figure}[!htb]
	\centering
	\includegraphics*[width=0.5\textwidth, height=0.4\textwidth]{img02.png}
	\caption{Timing System Configuration}
	\label{timing_system_configuration}
\end{figure}


\subsection{Event Generator}
\subsubsection{Event Generator Watch, evgWatch}\hspace*{\fill} \\
As shown Fig.~\ref{timing_data_structure}, evgWatch operates by receiving an external clock of 81.25 MHz (RF refernce clock) in the EVG mode. On the other hand, it operates by receiving on-board clock in the EVS mode. Only in the EVG mode, it synchronizes and decodes the IRIG-B signal from the GPS system. Arising the txAlarm signal at a cycle of 1 KHz, it generates the data of the stream data generator.\newline

\subsubsection{Stream Data Generator}\hspace*{\fill} \\
The stream data generator the timestamp and the count information of the EVG transmits in a 1 KHz cycle.
Table ~\ref{stream_data_packet} shows the sending data packet.

\begin{table}[h!t]
	%	\setlength\tabcolsep{3.8pt}
	\centering
	\caption{Stream Data Packet}
	\label{stream_data_packet}
	
	\begin{tabular}{@{}cll@{}}
		%\begin{tabular}{clc}}
		
		\hline
		\textbf{Number}	& \textbf{Data Packet}			& \textbf{Content}\\
		\hline
		0				& K-Code						& Header\\
		1				& 0xa7			                &\\
		\hline
		2				& Time0							& Year\\
		3				& Time1                         & Month\\
		4				& Time2                         & Day\\
		5				& Time3                         & Hour\\
		6				& Time4                         & Min,Sec\\
		7				& Time5                         & mSec\\								
		\hline
		8				& evTx Cnt0						& EVG Event Tx Count 0\\        
		9				& evTx Cnt1						& EVG Event Tx Count 1\\        
		10				& evTx Cnt2						& EVG Event Tx Count 2\\        
		11				& evTx Cnt3						& EVG Event Tx Count 3\\        
		12				& 0xff							& Tail\\        
		13				& 0x7a							& Tail\\        								
		14				& 0x7a							& Tail\\        								
		\hline
	\end{tabular}
\end{table}

\subsubsection{Event Trigger Source, evSrc}\hspace*{\fill} \\
The signals of DBus, evCodeA or evCodeB in the EVG are the SW user trigger, MXC 14EA, and the external inputs (TTL I/O module - low floor) 16EA. The bundle of these signals is defined as the event source (evSrc).

\begin{table}[h!t]
	\centering
	\caption{Event Source List}
	\label{event_tri_list}
	
	\begin{tabular}{@{}c|c|c|c@{}}
		%\begin{tabular}{clc}}
		\hline
		\textbf{Number}	& \textbf{Name}			& \textbf{Number}		& \textbf{Name}\\
		\hline
		0				& USER\_TRIGGER			& 15					& EXT\_IN\_SL\_00\\
		1				& MXC\_00				& 16					& EXT\_IN\_SL\_01\\
		2				& MXC\_01				& 17					& EXT\_IN\_SL\_02\\
		3				& MXC\_02				& 18					& EXT\_IN\_SL\_03\\
		4				& MXC\_03				& 19					& EXT\_IN\_SL\_04\\
		5				& MXC\_04				& 20					& EXT\_IN\_SL\_05\\
		6				& MXC\_05				& 21					& EXT\_IN\_SL\_06\\
		7				& MXC\_06				& 22					& EXT\_IN\_SL\_07\\
		8				& MXC\_07				& 23					& EXT\_IN\_SL\_08\\
		9				& MXC\_08				& 24					& EXT\_IN\_SL\_09\\
		10				& MXC\_09				& 25					& EXT\_IN\_SL\_10\\
		11				& MXC\_10				& 26					& EXT\_IN\_SL\_11\\
		12				& MXC\_11				& 27					& EXT\_IN\_SL\_12\\
		13				& MXC\_12				& 28					& EXT\_IN\_SL\_13\\
		14				& MXC\_13				& 29					& EXT\_IN\_SL\_14\\
		-				& -						& 30					& EXT\_IN\_SL\_15\\
		\hline
	\end{tabular}
\end{table}

\subsubsection{Ditributed Bus, DBus}\hspace*{\fill} \\
In Fig.~\ref{timing_data_structure}, the signal of the DBus through the timing network defines as 8 bits. Immediately, Its signals deliver to all EVRs without generating an event or calculating of any logic. This function is used to transmit the external input signal as soon as possible. Each bit of the DBus signal, which is one of the input sources, can be output by the enable bit.\newline

\subsubsection{Event Code A, evCodeA}\hspace*{\fill} \\
The evCodeA is a number code from 1 to 31 expressed in 5 bits. An input signal can be selected by a combination of the 5 bits. The evCodeA detects the rising edge of the input signal, then set the IRQ register and is cleared after outputting the evCodeA. If a signal is generated at the same time, the highest proiority is transmitted sequentially from 1 to 31. The mevCodeA, which is a signal mapping the evCodeA to the evCodeB, generates a configurable evCodeB from BRAM address 1 to 31 by a user.\newline

\subsubsection{Event Code B, evCodeB}\hspace*{\fill} \\
The evCodeB of 11 bits has a value of 0x001 to 0x7FF. The range of 0x7F0 to 0x7FF (128 codes) is used to the predefined functions. The evCodeB with the priority-based consists of seqA, seqB, mevCodeA, and evUser ($seqA > seqB > mevCodeA > evUser$). \newline

\subsubsection{Sequence A, B, seqA,B}\hspace*{\fill} \\
The sequence of the timing system is a function to transmit a sequence of the predefined evCodeB according to the preset time by the sequence trigger signal. The timing system provides two sequence procedures which are seqA and seqB. The sequence trigger signal can be generated with a combination of the input signals (evSrc). The sequence trigger mapped to each input signal is masked  by bitwise operation in the 31-bit register. A total of six sequence trigger combinations can be selected by masking the sequence trigger bit. The sequence A and sequence B sends the event code preset in the FPGA BRAM to the timing network according to the timestamp. 

\begin{figure}[!htb]
	\centering
	\includegraphics*[width=0.45\textwidth, height=0.35\textwidth]{img17.png}
	\caption{Sequence Finite State Machine}
	\label{sequence_fsm}
\end{figure}

The Fig. ~\ref*{sequence_fsm} shows the finite state machine implemented for the sequence A and B. The register set of the set\_evg\_enable transits the stop state to the standby state. By the sequence trigger signal, it enters the run state and reads data from the BRAM to generate the event code. If the register of the stop\_bit is 1, the run state terminates and the predefined seq\_repeatReg register value decreases. When the seq\_repeatReg register is 0, it becomes the pending state and transits to the standby state by the register value of the set\_evSeq\_Resume. If the seq\_repeatReg register value is set to 0, the sequence will be pended once. If it is set to 0xFFFF, the sequence will repeat indefinitely. It should be set the register of the set\_evg\_enable to disable the infinite loop. The maximum number of the event code B that can be set in Sequence is 2048. The data structure of a single record field has offset (0$\sim$2047), event code (0x1$\sim$0x7FF), stop bit (0,1), and timestamp (0x1$\sim$0xFFFFFFFF).

\subsection{Event Receiver}
\subsubsection{Event Receiver Watch, evrWatch}\hspace*{\fill} \\
The event receiver watch (evrWatch) decodes the stream data received from the timing network. It also provides the current time information to the system every 1 KHz.
The timing system generates the stream data in a certain pattern. If it does not match up the pattern, it is regarded as timing network reception error. The EVRs recover the clock synchronized with the external clock of EVG from the stream data and provides 1 PPS signal synchronized with the EVG.\newline

\subsubsection{Distributed Bus on Receiver }\hspace*{\fill} \\
The register of set\_evr\_enable can select whether it receives the distributed bus signals on EVRs. \newline

\subsubsection{Programmable Delayed Pulse, pdpOutput}\hspace*{\fill} \\
The register of set\_evr\_enable can select whether it receives the event code A or B. The event code B designates the address of the event mapping RAM. As a result, the programmable pulse trigger signals (pdpTrigger), which are the data bit decoded by the event code B of the corresponding address, are output. The pdptrigger 16 to 31 are selectively generated by the event code A or the decoded event code B. The pdpTrigger can be output the pulse width, delay, and polarity by the register of the cfg\_evr\_pdp.\newline

\subsubsection{Signal Output, exOut32}\hspace*{\fill} \\
The signal outputs (exOut32) of the event receiver is the final output sent to the device for control. In Table.~\ref*{evr_signal_output}, its source signal can be selected by the register of set\_evr\_pdpAndOut and cfg\_evr\_outSrcSel.

\begin{table}[h!t]
	\centering
	\caption{EVR Signal Output}
	\label{evr_signal_output}
	
	\begin{tabular}{@{}cll@{}}
		
		\hline
		\textbf{Output}	& \textbf{Source 1}				& \textbf{Source 2}\\
		\hline
		0 $\sim$ 7			& DBus 0 $\sim$ 7			& pdbOutput 0 $\sim$ 7\\
		8 $\sim$ 30			& pdpOutput 8 $\sim$ 30     \\
		31				& 1PPS evrOut					& pdbOutput 31\\        								
		\hline
	\end{tabular}
\end{table}

\subsubsection{Event Log, evLog}\hspace*{\fill} \\
The event log (evLog) stores the timestamp information when the values of the evCodeA, evCodeB, and pdpOutput outputs change. It is saved as 128bit data in total.\newline

\begin{itemize}
	\item MSB :	Day [9bit], Year [6bit], Hour [5bit], Min [6bit], Sec [6bit] [2b0], mSec [10bit], Clock tick [20bit][11b0], evCodeA [5bit], [5b0], evCodeB [11bit]
	\item LSB :	pdpOutput [32bit]
\end{itemize}

\subsection{Latency Calculation on Timing Network}
\begin{figure}[!htb]
	\centering
	\includegraphics*[width=0.45\textwidth, height=0.3\textwidth]{img15.png}
	\caption{Latency Time Calculation Mode}
	\label{latency_time_mode}
\end{figure}
To measure the latency of the optical cable length for the timestamp compensation (latency measurement mode), the UP-Link function is released. The cross point switch (CPS) setting of all devices except EVG is changed from FMC2 to FMC1. Then, the event fan-out of the timing system measures the delay time of each node by calculating the difference in data receiving time between GT0 and GT1 like the Fig.~\ref{latency_time_mode}.

\begin{figure}[!htb]
	\centering
	\includegraphics*[width=0.45\textwidth, height=0.23\textwidth]{img16.png}
	\caption{Latency Time Calculation}
	\label{latency_time}
\end{figure}
In Fig.~\ref{latency_time}, the latency time per node is the summation of AL and B.\\
\noindent
AL : Delay time proportional to the optical length\\
\space B  : Delay time for CPS and SFP Circuit \\
\space G  : Time of receiving EVG data from EVF \\
\space T1 : Delay time proportional to the optical length

\begin {align} \label{eqn_r}
R = G + 2(AL + B) \nonumber \\
R - G = 2(AL + B) 
\end {align}
R : Time received when EVF data was looped back in EVR \\
(R - G) : Time difference between G and R measured in FPGA
\begin {align*}
T1 = 2(A*1 + B) \\
T2 = 2(A*2 + B)
\end {align*}
T1 : Measuring R and G with 1x length of optical path \\
T2 : Measuring R and G with 2x length of optical path
\begin {align}
A = (T2 - T1)/2 \\
B = T1 - T2/2
\end {align}
In the equation (2) and (3), since the values of A and B are constants, the L value by the equation (1) can be estimated.
As a result, the latency time per node can be calculated.

\subsection{Timing System Configuration on Timing Network}

\begin{figure}[!htb]
	\centering
	\includegraphics*[width=0.45\textwidth, height=0.4\textwidth]{img14.png}
	\caption{Timing System Configuration}
	\label{timing_conf}
\end{figure}
The timing data generated from the EVG is distributed by the EVF. The FMC1 and FMC2 of the EVG can constitute an UP-Link connection. In the EVF, EVR and EVR-UP mode except EVG, the FMC1 SFP port is the timing data reception port and the FMC2 SFP of the EVF is the UP-Link connection port. The FMC2 SFP port of the EVR and EVR-UP can be used to expand the timing network.The above Fig.~\ref{timing_conf} shows the example configuration of the expandable timing network in which an UP-Link is connected through several EVFs.

\subsection{Firmware}

\subsection{Software}

\subsection{Timestamp Synchronization}


\hfill\break
TABLE \ref{sw-conf} shows the software modules that used to incorporate a distributed process control system and a parallel processing control system.

\begin{table}[h!t]
	%	\setlength\tabcolsep{3.8pt}
	\centering
	\caption{Software Modules}
	\label{sw-conf}
	
	\begin{tabular}{@{}lll@{}}
		%\begin{tabular}{clc}}
		
		\hline
		\textbf{Software} & \textbf{Contents}                 & \textbf{Module}     \\
		\hline
		EPICS        & Base R3.14.15.2                       & PS of Zynq            \\
		Libiio       & Industrial I/O library                &                       \\
		& supplied Analog Devices               &                       \\         
		Kernel       & Linaro kernel source                  &                       \\
		& including iio device driver           &                       \\         
		Bootloader   & U-Boot for zynq                       &                       \\
		IOC          & In-house using Libiio                 &                       \\
		
		\hline
		FPGA         & Analog Devices and                    & PL of Zynq            \\
		& In-house code                         &                       \\
		
		\hline
		Vivado       & Vivado14.4 including SDK              & Tool                  \\        
		& with node lock license                &                       \\            
		Busybox      & for rootfs system (free)              &                       \\
		Toolchain    & for ARM cross-compile (GNU)           &                       \\        
		\hline
		
	\end{tabular}
\end{table}


\section{Results}
The SDSoC (Software Defined System On Chip)\cite{sdsoc} environment is an Eclipse-based Integrated Development Environment (IDE) for implementing embedded systems using the zynq programmable SoC platform.The SDSoC environment includes support for the ZC702, ZC706, MicroZed, ZedBoard and Zybo development boards featuring the Zynq-7000 AP SoC.
\hfil\break\hfil\break
An SDSoC platform can define as follows:
\begin{itemize}
 	\item Hardware and Software Architecture
 	\item Application Context including Linux OS
 	\item Bootloader
 	\item Device Drivers for I/O or AXI Interface
 	\item Root File System
	\item External Memory Interface
	\item Custom Input/Output
	\item User-defined Libraries
\end{itemize}
A series of the works that were previously described in chapter [2,3] can be easily configured using the SDSoC platform as shown in Fig.\ref{sdsoc_control_system}. 

Fig.\ref{sdsoc_control_system} shows the configuration of the software platform to perform the application context using SDSoC on zynq PS, such as a linux kernel image, boot loader, root file system, device tree, and  the hardware platform to generate the FPGA logic through vivado on zynq PL. After this configuration, EPICS middleware that cross-compiled on the linux OS is interconnected with the C libraries through the SDSoC. Therefore it does not need to develope the AXI interface linux device driver.

\section{Conclusion}
Big scientific experiment facilities such as accelerators, nuclear fusion, and telescope are  required by the unity of the heterogenous distributed control system and the fast response  according to its characteristic. These requirements for the distributed and parallel environments must be satisfied at the same time. If the distributed control system using EPICS makes a connection with the high speed parallel processing of FPGA, it is possible to improve the performance and efficiency of the control system. Zynq SoC can be considered as an ideal device to satisfy with the distributed and high-speed parallel control system at the same time. More and more, the control environment of devices used in the big science experiment of the future will be used the distributed processing environment combined with the parallel processing environment. Therefore, it expect to be increased the requirements for the unified software abstraction layer to operate the heterogenous parallel proccessing hardwares."FPGA-based heterogeneous system (CPU + FPGA) using the OpenCL standard has a significant time-to-market advantage compared to traditional FPGA development using lower level hardware description languages (HDLs) such as Verilog or VHDL"\cite{opencl_on_altera}.  As a result, utilizing the OpenCL standard on the heterogeneous parallel processing hardwares may offer significantly higher performance and the unified abastract layer at much lower power. 



\begin{figure}[!htb]
	\centering
	\includegraphics*[width=90mm, height=70mm]{img03.png}
	\includegraphics*[width=90mm, height=70mm]{img04.png}
	\includegraphics*[width=60mm, height=50mm]{img17.png}
\end{figure}

% use section* for acknowledgement
\section*{Acknowledgment}
This work is supported by the Rare Isotope Science Project funded by Ministry of Science, ICT and Future Planning \textbf{(MSIP)} and National Research Foundation \textbf{(NRF)} of Korea (Project No. 2011-0032011).


% references section

% can use a bibliography generated by BibTeX as a .bbl file
% BibTeX documentation can be easily obtained at:
% http://www.ctan.org/tex-archive/biblio/bibtex/contrib/doc/
% The IEEEtran BibTeX style support page is at:
% http://www.michaelshell.org/tex/ieeetran/bibtex/
%\bibliographystyle{IEEEtran}
% argument is your BibTeX string definitions and bibliography database(s)
%\bibliography{IEEEabrv,../bib/paper}
%
% <OR> manually copy in the resultant .bbl file
% set second argument of \begin to the number of references
% (used to reserve space for the reference number labels box)
\begin{thebibliography}{3}

\bibitem{risp_overview}
Overview of the Rare Isotope Science Project of the Institute for Basic Science, Sunchan Jeong, New Physics: Sae Mulli, Vol. 66, No. 12, December 2016, pp. 1458~1464

\bibitem{risp} Y.~K.~Kwon, {\it et. al},``Status of Rare Isotope Science Project in Korea'',
Few-Body Syst 54, 961-966, (2013).


\bibitem{opencl}
OpenCL, https://www.khronos.org/opencl/

\bibitem{analog}
Analog Devices website, http://www.analog.com

\bibitem{linaro}
Linaro Document website, https://en.wikipedia.org/wiki/Linaro

\bibitem{u-boot}
U-Boot Document website, http://www.denx.de/wiki/U-Boot

\bibitem{busybox}
Busybox Document website, http://www.busybox.net/


\bibitem{iio}
Industrial I/O Document website, https://wiki.analog.com/resources/tools-software/linux-software/libiio

\bibitem{sdsoc}
SDSoC Document,\\ http://www.xilinx.com/support/documentation/sw\_manuals/xilinx2016\_1/ug1146-sdsoc-platforms-and-libraries.pdf

\bibitem{opencl_wiki}
OpenCL Definition, https://en.wikipedia.org/wiki/OpenCL

\bibitem{epics_v4}
EPICS Version 4, http://epics-pvdata.sourceforge.net/

\bibitem{epics_v4_normative}
EPICS Version 4 Normative Types,\\ http://epics-pvdata.sourceforge.net/docbuild/normativeTypesCPP/tip/\\documentation/ntCPP.html

\bibitem{opencl_on_altera}
OpenCL on Altera, https://www.altera.com/en\_US/pdfs/literature/wp/wp-01173-opencl.pdf

\end{thebibliography}

% that's all folks
\end{document}


