% !TeX spellcheck = de_DE
%\documentclass[11pt,a4paper]{article}
\documentclass[11pt
  , a4paper
  , article
  , oneside
%  , twoside
%  , draft
]{memoir}

\usepackage{control}
\usepackage[numbers]{natbib}


\begin{document}

\newcommand{\technumber}{
  RAON Control-Document Series\\
  Revision : v1.0,   Release : 2015-10-29 fixed date}
\title{\textbf{EPICS 설치 및 환경구성 자동화 스크립트}}

\author{이상일\thanks{silee7103@ibs.re.kr}\\
  Rare Isotope Science Project\\
  Institute for Basic Science, Daejeon, South Korea
}
\date{\today}

\renewcommand{\maketitlehooka}{\begin{flushright}\textsf{\technumber}\end{flushright}}
%\renewcommand{\maketitlehookb}{\centering\textsf{\subtitle}}
%\renewcommand{\maketitlehookc}{C}
%\renewcommand{\maketitlehookd}{D}

\maketitle

\begin{abstract}
RAON 제어 시스템은 Experimental Physics and Industrial Control System (EPICS)\cite{epics} 미들웨어를 사용하여 대형 이기종간의 장치들을 통합하여 하나의 제어시스템으로 통합한다. 본 문서는 EPICS 개발환경에 대한 표준구성을 자동화한 스크립트 사용을 명시한다. 
\end{abstract}

\hfil\break
\hfil\break
\hfil\break
\hfil\break
\chapter{EPICS 개발환경 표준화 구성}
\section{목적}
EPICS 표준화 구성을 통하여 각 장치간의 EPICS 개발환경(개발 디렉토리명, 라이브러리 표준, 개발환경변수 등) 및 운영 환경을 통일하고 그에 대한 형상을 맞추어 체계적인 형상관리를 이룰 수 있다. 소프트웨어 개발 및 운영에 있어 버전에 대한 형상은 매우 중요하며 또한 여러 장치별 개발 코드 및 업체들로 부터 개발된 소프트웨어의 코드 형상관리는 반드시 필요하다. EPICS 표준화 구성을 위하여 자동화 설치 스크립트가 개발되었으며 이 스크립트를 통해 EPICS 표준화 설치 구성에 대한 설명을 한다.


\section{요구사항}
자동화 설치 스크립트를 사용하기 전 다음 사항을 만족해야 한다.
\begin{itemize}
	\item Debian Linux 계열(whizzy,jessie) 32bit/64bit, Ubuntu Linux 32bit/64bit
	\item 설치될 EPICS Base R3.14.12.5
	\item Internet 접속 필요
	\item Git utility 설치 필요
\end{itemize}

Linux 설치가 완료되면 root 계정이 아닌 일반계정(예,ctrluser)을 생성한다. 이는 EPICS 환경 운영시 계정에 대한 컨트롤을 위함이다. 슈퍼유저 모드에서의 EPICS 설치 운영은 가급적 피하여야 한다. 아래 절차는 ctrluser 계정을 생성하여 진행한다.

\section{설치절차}
\subsection{Debian Package 관리 서버 등록}
관련 Package를 설치하기 전 Debian Package 관리 서버를 아래와 같은 두 라인을 추가 저장한다.
\begin{lstlisting}[style=termstyle]
root@silee:/home/ctrluser# vi /etc/apt/sources.list
# Insert following two lines
deb http://ftp.debian.org/debian/ jessie main contrib non-free
deb-src http://ftp.debian.org/debian/ jessie main contrib non-free
\end{lstlisting}

Package list를 update와 upgrade 한다.
\begin{lstlisting}[style=termstyle]
root@silee:/home/ctrluser# aptitude update && aptitude upgrade
\end{lstlisting}

\subsection{Git 설치}
Git은 소프트웨어 형상관리 목적으로 개발된 open source의 프로그램이다. 현재 제어그룹의 형상관리는 "github.com"\cite{github}을 이용한다. 아래와 같은 Package를 슈퍼유저 모드로 설치한다.

\begin{lstlisting}[style=termstyle]
root@silee:/home/ctrluser# aptitude install git
\end{lstlisting}

\subsection{scripts\_for\_epics download}
슈퍼유저 모드에서 빠져나와 다음과 같은 git clone을 진행한다. 해당 계정의 home에서 진행한다.
\begin{lstlisting}[style=termstyle]
ctrluser@silee:~$ git clone https://github.com/Sangil-Lee/scripts_for_epics
\end{lstlisting}
github의 Sangil-Lee\cite{github_sangil} 계정은 public 계정으로 암호입력 없이 scripts\_for\_epics 디렉토리가 다운로드 된다. 다운로드된 디렉토리로 이동하여 슈퍼유저 모드로 아래와 같은 절차를 따른다. 

\subsection{require\_package.sh}
"require\_packages.sh"은 EPICS 환경에 필요한 기본 extension package 및 환경변수 스크립트를 설치한다.
\begin{lstlisting}[style=termstyle]
ctrluser@:~$ cd scripts_for_epics/
ctrluser@:~/scripts_for_epics$ su
Password: 
root@:/home/ctrluser/scripts_for_epics# bash require_packages.sh all
.... package install....
root@:/home/ctrluser/scripts_for_epics# exit
\end{lstlisting}

\subsection{epics\_default\_installation.sh}
아래와 같이 해당 디렉토리로 이동 후 "epics\_default\_installation.sh"을 실행하여 EPICS Base 및 Extension을 설치한다. EPICS Base는 R3.14.12.5 버전이 설치된다. 해당 스크립트는 관련 package를 다운로드 받음과 동시에 컴파일을 진행하여 Binary 파일까지 생성한다.
\begin{lstlisting}[style=termstyle]
ctrluser@:~/scripts_for_epics$ bash epics_default_installation.sh 
... Download ... Compiling... Installation
\end{lstlisting}

\subsection{setEpicsEnv.sh}
컴파일 설치 후 아래와 같이 디렉토리 이동 후 환경변수를 로딩한다.
\begin{lstlisting}[style=termstyle]
ctrluser@:~/scripts_for_epics$ cd ../epics/R3.14.12.5/
ctrluser@:~/epics/R3.14.12.4$ ls
base  extensions  setEpicsEnv.sh
ctrluser@:~/epics/R3.14.12.4$ . setEpicsEnv.sh 
\end{lstlisting}
여기까지가 진행되면 EPICS 사용을 위한 기본 설치 구성이 완료된 상황이다.

\hfil\break
다음 절차부터는 EPICS 개발환경에 필요한 Synchronous Application 라이브러리 설치 및 RAON 제어그룹에서 개발 진행된 EPICS 개발 IOC 및 라이브러리 이다.

\subsection{epics\_synApps.sh}
EPICS IOC 개발에 매우 유용한 Asynchronous 개발 libraries 모음이다. 아래 절차를 실행하여 해당 소스의 다운로드 및 컴파일 설치를 완료한다.

 \begin{lstlisting}[style=termstyle]
 ctrluser@:~/epics/R3.14.12.4$bash ~/scripts_for_epics/epics_synApps.sh
 
 ctrluser@silee:~/epics/R3.14.12.5$ ls epicsLibs/synApps_5_8/support/
 alive-1-0            checkout.csh        export.csh                       mca-7-6            softGlue-2-4-3
 allenBradley-2-3     configure           ip-2-17                          measComp-1-1        sscan-2-10-1
 areaDetector-R2-0    dac128V-2-8         ip330-2-8                        modbus-2-7          std-3-4
 asyn-4-26            delaygen-1-1-1      ipac-2-13                        motor-6-9           stream-2-6a
 autosave-5-6-1       devIocStats-3-1-13  ipac.patch                       optics-2-9-3        update.bat
 busy-1-6-1           devIocStats.patch   ipUnidig-2-10                    propsetForHtml.csh  update.csh
 calc-3-4-2-1         dircmp.out          LICENSE                          quadEM-5-0          utils
 camac-2-7            documentation       love-3-2-5                       README              vac-1-5-1
 camac.patch          doMake              make_3.14.12.5_linux-x86_64.out  README.tmm          vme-2-8-2
 caputRecorder-1-4-2  dotag               make_3.15_linux-x86_64.out       seq-2-2-1           xxx-5-8-3
 checkout.bat         dxp-3-4             Makefile                         seq.patch
 \end{lstlisting}

\subsection{RAON siteApps/siteLibs Download}
다음 절차는 제어그룹에서 개발완료 또는 개발 중인 EPICS 개발 코드 디렉토리이다. 이는 github의 유료계정으로 형상관리를 진행하고 있다. 따라서 해당 코드를 다운로드하기 위하여는 먼저 github의 무료 계정을 생성한 후 제어그룹의 RaonControl 계정에 해당 계정의 인증절차를 거쳐야 한다.

따라서 본절은 향 후 SCL Demo 구성원의 EPICS training 시간에 자세한 설명이 이루어질 예정이며, 본 문서에서는 절차의 내용만 습득한다. 다운로드가 완료된 후 각 항목을 빌드하여 EPICS IOC를 구성한다. 아래는 제어그룹에서 개발한 siteLibs와 siteApps에 대한 디렉토리 구조를 보여준다. 해당 로컬 제어 EPICS IOC 개발 코드 및 업체의 개발 코드 역시 아래 개발 환경에 위치하여 형상관리가 진행되어야 한다.

\begin{lstlisting}[style=termstyle]
ctrluser@:~/epics/R3.14.12.4$ ls
base  extensions  setEpicsEnv.sh
ctrluser@:~/epics/R3.14.12.4$ git clone https://github.com/RaonControl/siteLibs
... After git account and password, downloading...

ctrluser@:~/epics/R3.14.12.4$ git clone https://github.com/RaonControl/siteApps
... After git account and password, downloading...


ctrluser@silee:~/epics/R3.14.12.5$ ls
base  epicsLibs  extensions  setEpicsEnv.sh  siteApps  siteLibs
ctrluser@silee:~/epics/R3.14.12.5$ ls siteApps/
abplc            exAsynRecord  gconpi         isol_mobiis   pvlists    README.md  snmpTest    Tr1  Tr5
bin              exSysMon      glassManPS     keithley6514  raspberry  rtp        sorensenPS  Tr2  Tr6
caGateway_kstar  exTimeStamp   grackApps      lsplcModbus   rdbMySQL   snmp2      srs725      Tr3  xgs600
ecrisApps        exTr5         integrateTest  mrfioc2       rdbPG      snmpMSU    streamSim   Tr4
ctrluser@silee:~/epics/R3.14.12.5$ ls siteLibs/
abplcLib_test  dbd            glassManPSLib  Makefile           README.siteLibs  snmpLib      timestampLib
bin            devlib2        ifstatLib      Makefile.template  RPiLibPack       snmpMSULib   XGS600LibSrc
configure      documentation  include        rdbpgLib           rtpLib           sysMonLib
db             ether_ipLib    lib            README             s7plcLib         TC353LibSrc
\end{lstlisting}




\chapter{Acknowledgement}
EPICS 표준화 구성에 대한 자동화 스크립트의 저자는 제어그룹의 이전 멤버였던 이정한 박사이며, 이 스크립트 작성에 대하여 감사의 뜻을 전합니다.

\clearpage

\bibliographystyle{unsrtnat}
\bibliography{./refs}

\end{document}

