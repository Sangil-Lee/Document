% !TeX spellcheck = en_US
%\documentclass[11pt,a4paper]{article}
\documentclass[11pt
  , a4paper
  , article
  , oneside
%  , twoside
%  , draft
]{memoir}

\usepackage{control}
\usepackage[numbers]{natbib}


\begin{document}

\newcommand{\technumber}{
  RAON Control-Document Series\\
  Revision : v1.0,   Release : 2016-03-14 fixed date}
\title{\textbf{FPGA 개발 툴 Vivado Floating License 서버 설치}}

\author{이상일\thanks{silee7103@ibs.re.kr} \\

  Rare Isotope Science Project\\
  Institute for Basic Science, Daejeon, South Korea
}
\date{\today}

\renewcommand{\maketitlehooka}{\begin{flushright}\textsf{\technumber}\end{flushright}}
%\renewcommand{\maketitlehookb}{\centering\textsf{\subtitle}}
%\renewcommand{\maketitlehookc}{C}
%\renewcommand{\maketitlehookd}{D}

\maketitle

\begin{abstract}
본 문서는 Xilinx FPGA 개발을 위하여 사용하는 Vivado 툴을 사용하기 위한 Floating License Server에 대한 구성 내용을 기술한다.
현재 제어그룹에서는 Vivado Floating License 2 Copies를 구매하여 사용 중이다.
\end{abstract}

FPGA 7 series 이상부터는 Vivado를 사용하여야 하며, Vivado를 사용하기 위한 라이센스 구성 방법은 크게 두가지로 나뉜다. 첫째는 FPGA 보드를 살 경우 그에 부여 되는 Node Lock License를 사용하는 방법이 있으며 둘째는 Floating License를 통하여 Activation 하는 방법이다. 첫째 방법은 "tech\_fpga\_dev" 매뉴얼에서 Node Lock License를 이용하여 설치하는 방법이 기술되어 있다. 여기서는 두번째 방법으로 설치하는 방법을 기술한다.




\chapter{라이센스 Software 구성 및 Activation License 요청}
먼저 Xilinx 사이트로 부터 로그인하여 라이센스 설치 파일을 다운로드 한 후 아래와 같은 절차를 수행한다.("linux\_flexlm.tar.gz")
License 설치 경로는 "/home/ctrluser/Vivado/Vivado/2014.4/bin" 디렉토리를 기준으로 수행한다.

\begin{lstlisting}[style=termstyle, escapechar=!]
It's possilble to access internet.

===Requesting======
1.Download license manager xilinx.com (silee7103, d~!1~!d)
2.use root or sudo
3.untar on "/home/ctrluser/Vivado/Vivado/2014.4/bin"
4.<Installed Dir="/home/ctrluser/Vivado/Vivado/2014.4/bin/linux_flexlm_v11.11.0_201410">/lnx64.o/install_fnp.sh
5.#lnx64.o>xlicsrvrmgr -p request.xml
-> This creates a license request into the trusted storage area which causes
XML and HTML file to be output containing your server's HostID inforamtion

6.Open the request.html file using web browser.
7.Your web browser should now open with a Xilinx login screen
8. Login and Account drop-down in Xilinx Site
9.Select the Activation license of your choice and click the "Activate Floating Lincense" button
10.A dialog will open and type 2 in "Requested Seats" and 0 in "Borrowed Seats": We have 2 copies floating license.
11. Click Next button
12. "Xilinx_License.xml" file will be e-mailed to you.
\end{lstlisting}

위의 절차 과정에서 9,10 항목은 중요 순서이다. 라이센스 파일을 구성하기 전 아래와 같이 lsb 모듈을 설치한다.

\begin{lstlisting}[style=termstyle, escapechar=!]
In Debian or Ubuntu, Should be installed lsb (linux standard base) package.
#>aptitude install lsb
\end{lstlisting}


\chapter{Software 구성 및 License 서버 구동}
1부터 3 항목을 실행 후 License File를 "filename.lic" 파일명으로 생성한다. 이때 입력하여야 할 세가지 항목은 host\_name, host\_id 및 port 번호이다. Port 번호는 디폴트 2100이 할당 되며, host\_name은 hostname command로 host\_id는 hostid command를 통하여 입력한다. 만들어진 lincense 파일은 해당 lmgrd.sh 스크립트가 존재한 곳에 복사를 하여 관리하며, 5번 항목과 같이 라이센스 서버를 구동한다. 
\begin{lstlisting}[style=termstyle, escapechar=!]
===Installing======
1.#lnx64.0>./xlicsrvrmgr -p Xilinx_License.xml
2. This will be stored the license information into your trusted storage area.
3.#lnx64.0>./xlicsrvrmgr -v "format=long"
4. Make license file (.lic)
SERVER <host_name> <host_id> <port>
USE_SERVER
VENDOR xilinxd

For hostname and hostid, you can get using "hostname" and "hostid" command
#>hostname
silee
#>hostid
010a4105

i.e)license.lic (IP address:10.1.5.65)
SERVER silee 010a4105 2100(default port)
USE_SERVER
VENDOR xilinxd

5. Run lmgrd (lincense manager)

i.e)
$lnx64.0>./lmgrd.sh -c license.lic(xilinx.lic) -l lmgrd.log
\end{lstlisting}

\section{Vivado Floating License 서버 정보}
현재 Vivado Floating License 서버를 구동하는 서버의 정보는 아래와 같다.

\begin{itemize}
	\item Server IP:10.1.5.65 
	\item Account:ctrluser, Password:q~4
	\item Hostname:silee(hostname command), Hostid:010a4105(hostid command)
	\item Path: /home/ctrluser/Vivado/Vivado/2014.4/bin/linux\_flexlm\_v11.11.0\_201410/lnx64.o
	\item License Filename: xilinx.lic
	\item Command: \$./lnx64.0>lmgrd.sh -c license.lic(xilinx.lic) -l lmgrd.log
\end{itemize}

\chapter{Client 설정}
MATLAB은 standalone 형태의 1user copy를 현재 개발 PC에 설정하였다. 또한, Vivado Floating License 원격서버를 Client에서 사용하기 위하여는 아래와 같은 MATLAB 환경변수(LM\_LICENSE\_FILE)와 Vivado 환경변수(XILINXD\_LICENSE\_FILE) 변수를 아래와 같이 설정한 후 Vivado에서 활성화된 라이센스를 확인한다.

\begin{lstlisting}[style=termstyle, escapechar=!]
===Vivado Client======
set environment variable,(in ~/.bashrc)
export LM_LICENSE_FILE=1700@localhost : -> MATLAB
export XILINXD_LICENSE_FILE=2100@silee(10.1.5.65) : -> Vivado

$>vivado

Check "Help > Manage License..." Menu
****Activation Based Licenses: Permanent ****
Analyzer
HLS
Implementation
SDK
Simulation
Systhesize
SysGen
VIVADO_HLS
Vivado_System_Edition
*********************************************
\end{lstlisting}

\clearpage
\bibliographystyle{unsrtnat}
\bibliography{./refs}

\end{document}

