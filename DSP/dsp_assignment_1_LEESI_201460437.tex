% !TeX spellcheck = en_US
%\documentclass[11pt,a4paper]{article}
\documentclass[11pt
  , a4paper
  , article
  , oneside
%  , twoside
%  , draft
]{memoir}

\usepackage{control}
\usepackage{kotex}
\usepackage[numbers]{natbib}


\begin{document}

\newcommand{\technumber}{
  Digital Signal Processing using MATLAB\\
  Document 1: 2016-03-19}
\title{\textbf{Digital Signal Processing: 실습 1 \\
		제1장 matlab 사용법 \\ ( matlab tutorials for signal processing)}}

\author{이상일\thanks{silee7103@ibs.re.kr} \\

  학번: 201460437\\
  Computer Engineering, Chungnam National University 
}
\date{\today}

\renewcommand{\maketitlehooka}{\begin{flushright}\textsf{\technumber}\end{flushright}}
%\renewcommand{\maketitlehookb}{\centering\textsf{\subtitle}}
%\renewcommand{\maketitlehookc}{C}
%\renewcommand{\maketitlehookd}{D}

\maketitle

\begin{abstract}
MATLAB을 사용한 Digital Signal Processing에 대한 실습과제에 대한 Documents를 구성한다.
\end{abstract}

\chapter{문제 1:\\Solve Example 1.1 by using matlab in section 1.2.2. Implement three approaches.}

\begin {equation}
 x(t) = \sin (2*\pi*t) + \frac{1}{3}\sin(6*\pi*t) + \frac{1}{5}\sin(10*\pi*t) = \displaystyle\sum_{k=1}^{3} \frac{1}{3}\sin(2*\pi*t), 0 \leq t \leq 1
\end {equation}




\section{첫번째 방법: }
세개의 정현파를 위한 루프 카운트 k와 샘플링 카운트 t에 대한 각각을 이중 루프를 이용하여 구현한다.(첨부파일 TestLoop1.m)

\begin{lstlisting}[style=termstyle]
t=0:0.01:1; N=length(t); xt=zeros(1,N); % N elements xt row vector
for n = 1:N
	tmp = 0;
		for k=1:3  %3 elements sin function loop
			tmp = tmp + (1/k)*sin(2*pi*k*t(n));
		end
	xt(n) = temp;
end
\end{lstlisting}
\clearpage

\section{두번째 방법: }
루프연산을 줄이기 위한 방법으로 Vector의 원소를 이용하는 대신 Vector 연산을 직접한다.(첨부파일 TestLoop2.m)

\begin{lstlisting}[style=termstyle]
t=0:0.01:1; xt1=zeros(1,length(t));
for k=1:3
	xt1 = xt1 + (1/k)*sin(2*pi*k*t);
end
\end{lstlisting}


\section{세번째 방법: }
행 Vector와 행렬 곱셈 연산을 이용한다.(첨부파일 TestMatrix.m)

\begin{lstlisting}[style=termstyle]
t=0:0.01:1; k=1:3;
xt2 = (1./k)*sin(2*pi*k'*t);
\end{lstlisting}


\chapter{문제 2: \\Implement SCRIPTS AND FUNCTIONS of matlab in Section 1.2.3}

\section{Scripts 이용 방법}
일련의 interactive matlab 사용에 대한 batch 형태의 프로그램을 사용하는 방법이며 .m 파일 확장자를 사용한다.(첨부파일 TestScript.m)
스크립트에서 사용되는 변수는 전역변수로 설정된다.
\begin{lstlisting}[style=termstyle]
t = 0:0.01:1; k = 1:2:5; ck = 1./k;
xt = ck * sin(2*pi*k'*t);
\end{lstlisting}

\section{Function 이용 방법}
스크립트와 동일하게 .m 파일 확장자를 사용하며, 내부에서 사용되는 변수는 로컬 스택변수로 사용되며, 외부호출을 통하여 사용될 수 있다.(첨부파일 TestFun.m)
\begin{lstlisting}[style=termstyle]
function xt = TestSumFun(t, ck)
K = length(ck); k = 1:K;
ck = ck(:)';  t = t(:)';
xt= ck*sin(2*pi*k'*t);
\end{lstlisting}
위의 작성된 함수는 아래와 같이 함수 호출과정을 통하여 사용될 수 있다. 주의 사항으로는 작성된 함수는 "함수명.m" 로 작성되어야 한다. 
\begin{lstlisting}[style=termstyle]
>> t=0:0.01:1;
>> ck=1:2:5;
>> xt=TestFun(t,ck);
>> xt
xt =
Columns 1 through 13
0    1.3757    2.7120    3.9709    5.1167    6.1175    6.9459    7.5802    8.0049    8.2109    8.1962    7.9658    7.5312
Columns 14 through 26
6.9102    6.1261    5.2073    4.1855    3.0953    1.9729    0.8545   -0.2245   -1.2310   -2.1350   -2.9107   -3.5374   -4.0000
Columns 27 through 39
-4.2894   -4.4028   -4.3437   -4.1215   -3.7512   -3.2528   -2.6502   -1.9706   -1.2435   -0.4991    0.2324    0.9220    1.5430
Columns 40 through 52
2.0721    2.4899    2.7819    2.9389    2.9572    2.8386    2.5907    2.2262    1.7621    1.2199    0.6237    0.0000   -0.6237
Columns 53 through 65
-1.2199   -1.7621   -2.2262   -2.5907   -2.8386   -2.9572   -2.9389   -2.7819   -2.4899   -2.0721   -1.5430   -0.9220   -0.2324
Columns 66 through 78
0.4991    1.2435    1.9706    2.6502    3.2528    3.7512    4.1215    4.3437    4.4028    4.2894    4.0000    3.5374    2.9107
Columns 79 through 91
2.1350    1.2310    0.2245   -0.8545   -1.9729   -3.0953   -4.1855   -5.2073   -6.1261   -6.9102   -7.5312   -7.9658   -8.1962
Columns 92 through 101
-8.2109   -8.0049   -7.5802   -6.9459   -6.1175   -5.1167   -3.9709   -2.7120   -1.3757   -0.0000
>> 
\end{lstlisting}

\chapter{문제 3: \\By using basic plotting command, generate plots of simple sinusoidal wave satisfying options in section 1.2.4.}

\section{plot 이용 방법}
연속적인 시간에 따른 신호를 2D 형태로 도식화하기 위한 방법이며, 아래와 같이 표시 할 수 있다.
\begin{lstlisting}[style=termstyle]
t=0:0.01:2;
x=sin(2*pi*t);
plot(t, x, 'b');
xlabel('t in sec'); ylabel('x(t)');
title ('Plot of sin(2\pi, t)');

\end{lstlisting}

\section{stem 이용 방법}
연속적인 시간과는 달리 이산시간에 대한 count index를 가지고 신호를 도식화 한다. sampling 형태의 신호를 표시하기에 적합하다.
\begin{lstlisting}[style=termstyle]
n=0:1:40;
x=sin(0.1*pi*n);
Hs=stem(n,x,'b','filled');
set(Hs,'markersize',4);
xlabel('n'); ylabel('x(n)');
title ('Stem Plot of sin(0.2 pi n)');
\end{lstlisting}

\clearpage

\chapter{문제 4: \\Applications of digital signal processing in section 1.3. Implement three application by using matlab. }
\begin{lstlisting}[style=termstyle]
load handel;
sound (y,Fs); pause(10);
\end{lstlisting}
MATLAB에 내장된 handel 음원 파일을 로딩 후 10초간 sound 함수를 통하여 음원을 기본 sampling 한다.
아래와 같이 Echo 음을 생성, 제거 시킨다.

\hfill\break

\section{Echo Generation: }
\begin {equation}
x[n] = y[n] + \alpha y[n-D], \mid \alpha \mid < 1
\end {equation}
Echo 생성은 식(2)와 같은 차등방정식이 적용되며, 아래와 같은 filter 함수를 사용하여 x[n]을 생성한다.

\begin{lstlisting}[style=termstyle]
alpha=0.9; D=4196;
b = [1, zeros(1,D), alpha];
x = filter(b,1,y);
sound(x, Fs);
\end{lstlisting}

\section{Echo Removal: }

\begin {equation}
w[n] + \alpha w[n-D] = x[n]
\end {equation}

\begin{lstlisting}[style=termstyle]
w = filter(1, b, x);
sound(w,Fs)
\end{lstlisting}

\section{Digital Reverberation: }
잔향에 대한 digital 신호 처리는 아래와 같은 차분 방정식을 적용한다.
\begin {equation}
\displaystyle\sum_{k=0}^{N-1} \alpha^k y[n-kD]
\end {equation}

\begin {equation}
x[n] = \alpha y[n] + y[n-D] + \alpha x[n-D], \mid \alpha \mid < 1
\end {equation}

\end{document}

